\documentclass[main.tex]{subfiles}

\begin{document}

\section*{Abbildungen und Funktionen}
\begin{karte}{Schubfachprinzip und Folgerungen}
    Ist \( \abb{f}{[m]}{[n]} \) eine Funktion mit 
    \( m > n \), so existiert ein \( k \in [n] \) mit 
    \( f^{-1}(\set{k}) \supseteq \set{m_1, m_2}  \), 
    wobei \( m_1 \neq m_2 \).\\
    Sind \( n,m \in \N \) und ist \( \abb{f}{[m]}{[n]} \) injektiv 
    \( \Rightarrow m \geq n \).
    Sind \( n,m \in \N \) und ist \( \abb{f}{[m]}{[n]} \) bijektiv 
    \( \Rightarrow m = n \).
\end{karte}
\begin{karte}{Mächtigkeit von Mengen}
    Seien \( M, A, B \) Mengen.
    \begin{itemize}
        \item \( M \) heißt \textit{endlich} \( \Leftrightarrow 
        M = \emptyset \) oder \( \exists n \in \N \, \exists 
        \abb{f}{[n]}{M} \) bijektiv.
        \item \( M \) heißt \textit{unendlich} 
        \(\Leftrightarrow M \) ist nicht endlich 
        \item \( A, B \) heißen gleichmächtig \( \Leftrightarrow 
        \exists \abb{f}{A}{B} \) bijektiv.
        \item \(M\) heißt \textit{abzählbar} 
        \( \Leftrightarrow \exists \abb{f}{\N}{M} \) bijektiv
        \item \( M \) heißt \textit{überabzählbar} \( \Leftrightarrow 
        M \) ist unendlich und nicht abzählbar.
    \end{itemize}
    \( \Q \) ist abzählbar und \( \R \) ist überabzählbar.
\end{karte}
\begin{karte}{Satz von Cantor und Berenstein}
    Seien \(A, B \) Mengen und \( \abb{f}{A}{B} \) 
    sowie \( \abb{g}{B}{A} \) injektiv.
	Dann existiert eine Bijektion \( \abb{h}{A}{B} \).
\end{karte}
\begin{karte}{Vereinigung abzählbarer Mengen}
    \begin{enumerate}
		\item Jede Teilmenge einer abzählbaren Menge ist abzählbar.
        \item Für alle \(j \in \N \) sei \(A_j \) 
        eine abzählbare Menge. Dann ist 
        \[ \bigcup_{j \in \N}{A_j} \] abzählbar.
	\end{enumerate}
\end{karte}
\begin{karte}{Definition Binomialkoeffizient und Rekursionsformel}
    Seien \(\alpha \in \R, k \in \N \).
    \[ \binom{\alpha}{k} \Leftrightarrow 
    \frac{ \alpha(\alpha-1)\cdots(\alpha-k+1) }{k!},
    \quad \binom{\alpha}{0} \Leftrightarrow 1 \]
    heißt Binomialkoeffizient von \(\alpha \) und \(k \). \\
    Seien \(\alpha \in \R, k \in \N \). Dann gilt
    \[ \binom{\alpha + 1}{k} = \binom{\alpha}{k} 
    + \binom{\alpha}{k-1}. \]
\end{karte}
\begin{karte}{Binomische Formel}
    Seien \(a, b \in \R, n \in \N \). Dann gilt
	\[ {(a+b)}^n = \sum_{k=0}^{n} \binom{n}{k} a^k b^{n-k}. \]
\end{karte}
\begin{karte}{Mächtigkeit der Potenzmenge}
    Sei \(A \) eine endliche Menge. Dann gilt 
    \( \# \PO(A) = 2^{\#A} \).\\
    Es existiert keine surjektive Funktion 
    \(\abb{f}{A}{\PO(A)} \).
\end{karte}
\end{document}