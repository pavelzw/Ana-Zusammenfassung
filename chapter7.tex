\documentclass[main.tex]{subfiles}

\begin{document}
\section*{Folgen}

\begin{karte}{Definition Folge}
    Sei \( X \neq \emptyset \) eine Menge. Eine Folge in 
    \(X \) oder auch \(X \)-wertige Folge ist eine Funktion 
	\[ \abb{f}{\N}{X}, n \mapsto f(n) \in X. \]
	Wir setzen \(a_n := f(n), \, n \in \N \).\\
    \(a_n \) heißt \(n \)-tes Folgenglied. 
    Wir schreiben auch 
    \( {(a_n)}_{n \in \N} \) oder kurz \( {(a_n)}_n \).\\
    Ist \( X = \R \), so heißt die Folge reellwertig oder 
    reelle Folge. \\
	Dabei gilt \( {(a_n)}_{n\in\N} \subset \R \).
\end{karte}
\begin{karte}{Konvergenz reeller Folgen}
    Eine reelle Folge \( {(a_n)}_{n \in \N} \) konvergiert 
    gegen ein \(a \in \R \), falls 
    \[ \forall \, 
    \varepsilon > 0 \, \exists \, k \in \N : \; 
    \forall \, n \geq k \text{ folgt } \abs{a_n - a} 
    < \varepsilon. \]
    Die Zahl \(a\) heißt Grenzwert der Folge, \\
    wir schreiben \( \limes{n} a_n = a \) oder 
    \( a_n\rightarrow a \) (für \( n\rightarrow\infty \)).\\
    Eine (reelle) Folge heißt konvergent, 
    falls ein \( a\in\R \) der Grenzwert 
    der Folge ist, andernfalls heißt die Folge divergent.\\
    
    Falls die reelle Folge \({(a_n)}_{n\in\N}\) konvergiert, 
    so ist ihr Grenzwert eindeutig bestimmt.
\end{karte}
\begin{karte}{Beschränktheit von Folgen}
    Eine Folge \( {(a_n)}_n \subset \R \) heißt 
    beschränkt, wenn für \\
    \( C\geq 0 \) gilt \( \abs{a_n} \leq C \; \forall n\in\N \).
    \begin{itemize}
        \item nach oben beschränkt, wenn es ein \(C\in\R \) gibt mit 
        \( a_n\leq C \, \forall n\in\N \).
        \item nach unten beschränkt, wenn es ein \(C\in\R \) gibt mit 
        \( a_n\geq C \, \forall n\in\N \).
    \end{itemize}
    Jede konvergente Folge ist beschränkt.
\end{karte}
\begin{karte}{Rechenregeln für Grenzwerte}
    Es gelte \(a_n\rightarrow a, b_n \rightarrow b\) 
    für \(n\rightarrow\infty \).
	\begin{enumerate}
        \item \(\forall \lambda, \mu \in\R \) ist 
        \( {(\lambda a_n + \mu b_n)}_{n\in\N} \) 
        konvergent mit Grenzwert 
		\[\limes{n} (\lambda a_n + \mu b_n) = \lambda a + \mu b.\]
        \item Die Folge \( {(a_n b_n)}_{n\in\N} \) 
        konvergiert mit Grenzwert 
		\[\limes{n} {(a_n b_n)} = ab.\]
        \item Falls \(b\neq 0\), so gibt es ein 
        \( K_0 \in\N \) mit \(b_n \neq 0 \, \forall 
        \, n\geq K\) und die Folge 
        \( {\left(\frac{a_n}{b_n}\right)}_{n\geq K_0}\) 
        ist konvergent mit Grenzwert 
		\[\limes{n} \frac{a_n}{b_n} = \frac{a}{b}.\]
	\end{enumerate}
\end{karte}
\begin{karte}{Sandwichlemma}
    Seien \( {(a_n)}_n, {(b_n)}_n\) konvergente 
    reelle Folgen, \( a_n\rightarrow a, 
    b_n \rightarrow b, n\rightarrow\infty \). Dann gilt
	\begin{enumerate}
            \item aus \(a_n\leq b_n\) für fast alle \(n\) 
            folgt \(a\leq b\).
            \item sind \(c,d\in\R, c\leq a_n\leq d \) für fast alle 
            \(n \Rightarrow c\leq a\leq d\)
            \item (Sandwichlemma) Ist \(a_n \leq c_n \leq b_n \) 
            für fast alle \(n\) (\( {(c_n)}_n \) weitere reelle Folge) 
            und \( a=b \Rightarrow {(c_n)}_n \) konvergiert und 
            \( \lim\limits_{b\rightarrow\infty}c_n = a \) (\( =b \)).
	\end{enumerate}
\end{karte}
\begin{karte}{Uneigentliche Konvergenz}
    Die Folge \( {(a_n)}_n \) konvergiert 
    uneigentlich (divergiert bestimmt) 
    gegen \(+\infty \), falls 
    \[ \forall R>0 \;\exists K\in\N 
    \text{ mit } a_n >R \quad \forall n\geq K. \]
    Schreiben \( \limes{n} a_n = \infty \) 
    oder \( a_n \rightarrow +\infty, n\rightarrow \infty \)
	Analog für \( \limes{n} a_n = -\infty \), falls 
	\[ \forall R<0 \;\exists K\in\N: a_n < R \;\forall n\geq K. \]
\end{karte}
\begin{karte}{Kehrwerte von Folgen}
    \begin{enumerate}
        \item Aus \( \abs{a_n} \rightarrow \infty, 
        n\rightarrow\infty \) folgt \(\frac{1}{a_n} \rightarrow 0,
        n\rightarrow\infty \).
        \item Aus \( a_n\rightarrow 0, a_n > 0 \) 
        (bzw.\  \(a_n<0\)) \( \forall n \) folgt 
        \( \frac{1}{a_n} \rightarrow\infty, 
        n\rightarrow\infty \) \\
        (\( \frac{1}{a_n} 
        \rightarrow -\infty, n\rightarrow\infty \)).
	\end{enumerate}
\end{karte}

\end{document}