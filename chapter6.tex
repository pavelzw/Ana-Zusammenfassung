\documentclass[main.tex]{subfiles}

\begin{document}

\begin{karte}{\(n\)-te Wurzel}
    Sei \(\alpha \in \R, a>0 \) und \(n \in \N \). 
    Dann existiert die \(n\)-te Wurzel von \(a\) 
    als eindeutige reelle Zahl, d.\ h.\  
    \(\existse \, \alpha \in \R \) mit 
    \(\alpha > 0 \) und \(\alpha^n = a \).
\end{karte}

\end{document}