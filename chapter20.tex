\documentclass[main.tex]{subfiles}

\begin{document}
\section*{Funktionenfolgen}
\begin{karte}{Definition punktweise Konvergenz}
    \( f_n : D \rightarrow \C \) konvergiert punktweise 
    gegen \( f: D \rightarrow \C \), falls 
    \[ \limes{n} f_n(x) = f(x) \; \forall \, x \in D. \]
\end{karte}
\begin{karte}{Definition beschränkte Funktion}
    \( f: D \rightarrow \C \) heißt beschränkt, falls 
    \[ \underset{x\in D}{\sup} \abs{f(x)} < \infty \]
\end{karte}
\begin{karte}{Supremumsnorm}
    Für \( f\in B(D,\R) \) (oder \( B(D,\C) \)) setzen wir 
    \[ ||f||_\infty := \underset{x\in D}{\sup} \abs{f(x)}\]
    hat alle Eigenschaften einer Norm.
    \begin{enumerate}
        \item \( ||f||_\infty \geq 0 \) und 
        \( ||f||_\infty = 0 \Leftrightarrow f = 0 \) 
        (Nullfunktion)
        \item \( ||\lambda f||_\infty = |\lambda| \cdot 
        ||f||_\infty \) (homogen), \( \lambda \in \R \)
        \item \( ||f+g||_\infty \leq ||f||_\infty 
        + ||g||_\infty \) (Dreiecksungleichung) \\
        für \( f, g \in B(D, \R) \) (bzw. \( B(D,\C) \))
    \end{enumerate}
\end{karte}
\begin{karte}{Definition gleichmäßige Konvergenz}
    Eine Folge \( \abb{f_n}{D}{\R} \) (oder \( \C \)) 
    konvergiert gleichmäßig gegen \( f: D \rightarrow \R \), 
    falls 
    \[ \limes{n} ||f_n - f ||_\infty = 0. \]
\end{karte}
\begin{karte}{lokal gleichmäßige Konvergenz von Potenzreihen}
    Ist \(0 < R < \infty \) der Konvergenzradius der Potenzreihe 
    \(P(z) = \sum_{n=0}^{\infty} a_n z^n \), so konvergiert
    \(P(k) = \sum_{l=0}^{k} a_l z^l \) gleichmäßig auf
    \([-R + \delta, R - \delta] \) gegen \( P \) für alle 
    \( 0 < \delta < R \)
    \[ \underset{\abs{z} \leq R-\delta}{\sup} \abs{P(z)-P_k(z)}
    \overset{k \rightarrow \infty}{\longrightarrow} 0 \]
    Ist \( R = \infty \), so konvergiert \( P_k \) auf 
    \( [-L,L] \) gleichmäßig gegen \( P \) für jedes feste
    \( L > 0 \) (lokal gleichmäßige Konvergenz)
\end{karte}
\begin{karte}{Gleichmäßige Konvergenz und Stetigkeit}
    Seien \( f_n : D \rightarrow \C, n\in\N \) eine 
    stetige Funktionenfolge auf \( D \subset \R \) 
    (oder \( D = \C \)), die gleichmäßig gegen 
    \( f: D \rightarrow \C \) konvergiert, d.\ h.\ 
    \[ ||f_n - f||_\infty \rightarrow 0 
    \text{ für } n\rightarrow\infty. \]
    Dann ist \(f\) auch stetig auf \(D\).
\end{karte}
\begin{karte}{Stetigkeit von Potenzreihen}
    Sei \( P(z) = \sum_{n=0}^\infty a_n z^n \) eine komplexe 
    Potenzreihe mit \( R > 0 \). Dann ist \( P \) stetig auf 
    der Kreisscheibe 
    \( B_R(0) = \set{ z\in\C \; \vert \; \abs{z} < R } \).
\end{karte}
\begin{karte}{Vertauschen von Konvergenz und Ableitung}
    Seien \( f_n : I \rightarrow \R \) (oder \( \C \)) stetig 
    und differenzierbar auf \( I = (a,b) \subset \R \).
    Es gelte 
    \begin{enumerate}
        \item \( \set{f_n(x_0)}_{n\in\N} \) konvergiert für 
        \( x_0 \in I \).
        \item Die Folge der Ableitungen 
        \( \set{ f_n' }_{n \in \N} \) 
        konvergiert gleichmäßig gegen eine Funktion 
        \( g: I \rightarrow \R \) (oder \( \C \)), also 
        \( ||f_n' - g||_\infty 
        \overset{n\rightarrow\infty}{\longrightarrow} 0 \).
        Dann konvergiert \( \set{f_n}_n \) gleichmäßig gegen 
        \( f: I \rightarrow \R \) (oder \( \C \)). \\
        \( f \) ist differenzierbar und \( f' = g \).
    \end{enumerate}
    Konvergenz ist lokal gleichmäßig falls 
    \[ a = -\infty \text{ oder } b = \infty. \]
\end{karte}
\end{document}