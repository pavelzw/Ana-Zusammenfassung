\documentclass[main.tex]{subfiles}

\begin{document}
\section*{Satz von Rolle, MWS, Extrema}
\begin{karte}{Definition lokale Extrema}
    Die Funktion \( f : [a,b] \rightarrow \R \) (oder \( (a,b) \))
    hat in \( x_0 \in [a,b] \) ein lokales Minimum (oder in 
    \( x_0 \in (a,b) \)), falls \( \exists \, \delta > 0 \), 
    sodass 
    \[ f(x_0) \leq f(x) \, \forall \, x \in U_\delta(x_0)
    \cap [a,b] = (x_0 - \delta, x_0 + \delta) \cap [a,b] \]
    Ist \( x_0 \in (a,b) \), so kann \( \delta > 0 \) so klein 
    gewählt werden, dass \( (x_0 - \delta, x_0 + \delta) 
    \cap [a,b] = (x_0 - \delta, x_0 + \delta) \). \\
    Ist sogar \( f(x_0) < f(x) \,\forall \, x \in 
    (x_0 - \delta, x_0 + \delta) \cap [a,b] \), so heißt 
    das lokale Minimum isoliert (oder strikt).\\
    Analog: \( f \) hat in \( x_0 \) ein (indirektes) lokales 
    Maximum, falls \(-f\) in \(x_0\) ein (isoliertes) lokales 
    Minimum hat.\\
    Lokale Extrema \(=\) lokale Minima oder Maxima.
\end{karte}
\begin{karte}{Notwendige Bedingung für Extrema}
    \( f: (a,b) \rightarrow \R \) habe in \(x_0 \in (a,b) \) 
    ein lokales Extremum. Ist \(f\) in \(x_0\) differenzierbar, 
    so gilt \( f'(x_0) = 0 \)
\end{karte}
\begin{karte}{Satz von Rolle}
    Sei \( f: [a,b] \rightarrow \R \) stetig und differenzierbar 
    in \( (a,b) \) mit \( f(a) = f(b) = 0 \).
    \[ \Rightarrow \exists \, \xi \in (a,b) : f'(\xi) = 0. \]
\end{karte}
\begin{karte}{Mittelwertsatz}
    \( f: [a, b] \rightarrow \R \) stetig und differenzierbar 
    auf \((a,b)\).
    \[ \Rightarrow \exists \, \xi \in (a,b): f'(\xi) = \frac{f(b)-f(a)}{b-a} \]
\end{karte}
\begin{karte}{Monotoniekriterium}
    Sei \( f : [a,b] \rightarrow \R \) stetig und differenzierbar 
    auf \( (a,b) \).\\
    Dann gilt:
    \begin{align*}
        f'(x) = 0 \,\forall \, x \in (a,b) \Leftrightarrow 
        f \text{ ist konstant auf } [a,b].\\
        f'(x) \geq 0 \,\forall \, x \in (a,b) \Leftrightarrow 
        f \text{ ist wachsend auf } [a,b].\\ 
        f'(x) \leq 0 \,\forall \, x \in (a,b) \Leftrightarrow 
        f \text{ ist fallend auf } [a,b].\\
    \end{align*}
    Strikte Ungleichungen implizieren strenge Monotonie auf 
    \( [a,b] \).
\end{karte}
\begin{karte}{Schrankensatz}
    Sei \( f: [a,b] \rightarrow \R \) stetig und differenzierbar 
    auf \( (a,b) \).\\
    Dann gilt für \( a \leq x_1 < x_2 \leq b \): 
    \begin{enumerate}
        \item \( f'(x) \geq m \,\forall \, x \in (a,b) 
        \Rightarrow f(x_2) - f(x_1) \geq m(x_2 - x_1) \).
        \item \( f'(x) \leq M \, \forall \, x\in (a,b) 
        \Rightarrow f(x_2) - f(x_1) \leq M(x_2 - x_1) \).
    \end{enumerate}
\end{karte}
\begin{karte}{Definition Lipschitzstetigkeit}
    Sei \( f: [a,b] \rightarrow \R^d \) stetig und 
    differenzierbar auf \( (a,b) \). \\
    \( f \) ist Lipschitz-stetig, wenn ein \(L\) 
    existiert, sodass 
    \[ \abs{ f(x_1) - f(x_2) } \leq L \abs{x_1 - x_2}. \]
    Gilt \( ||f'(x)|| \leq L \, \forall \, x \in (a,b) \), 
    so folgt 
    \[ ||f(x) - f(y)|| \leq L ||x-y|| 
    \; \forall \, x,y \in [a,b] \]
\end{karte}
\begin{karte}{Definition \(k\)-te Ableitung}
    Die \( k \)-te Ableitung von 
    \( f: (a,b) \rightarrow \R \) in \( x_0 \) ist 
    induktiv definiert durch 
    \[ f^{(k)}(x_0) = (f^{(k-1)})'(x_0). \]
    Damit \( f^{(k)}(x_0) \) definiert ist, müssen 
    also die Ableitungen bis \( (k-1) \)-ter Ordnung 
    in einer Umgebung von \( x_0 \) definiert sein und 
    \( f^{(k-1)} \) muss in \( x_0 \) differenzierbar sein.
\end{karte}
\begin{karte}{Lokale Extrema erkennen}
    Sei \( f: (a,b) \rightarrow \R \) differenzierbar. \\
    Ist \( x_0 \in (a,b) \) und gelte 
    \[ f'(x_0) = 0 \text{ und } f''(x_0) \text{ existiert.} \]
    Dann gilt:
    \begin{enumerate}
        \item Ist \( f''(x_0) > 0 \), so hat \( f \) in 
        \( x_0 \) ein isoliertes lokales Minimum.
        \item Hat \( f \) in \( x_0 \) ein lokales 
        Minimum, so folgt \( f''(x_0) \geq 0 \).
    \end{enumerate}
    Analoge Aussagen mit umgekehrten Ungleichungen gelten 
    für Maxima.
\end{karte}
\end{document}