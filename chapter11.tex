\documentclass[main.tex]{subfiles}

\begin{document}

\section*{Reihen}
\begin{karte}{Beschränktheit von Reihen}
    Sind \(a_n \geq 0, n\in\N \), dann gilt\\
    entweder ist \(S_n = \sum_{k=1}^{n}a_n\) 
    nach oben beschränkt und dann ist 
    \[ \sum_{k=1}^{\infty}a_k = \limes{n} 
    S_n \in [0,\infty) \]
    oder \(S_n \rightarrow \infty, 
    n\rightarrow\infty \) und dann 
    ist \[ \sum_{k=1}^{\infty}a_k 
    = \limes{n}S_n = +\infty. \]
    Sind \(a_n \geq 0, n\in \N \), so ist
	\begin{align*}
        \text{entweder } 
        &\sum_{n=1}^{\infty}a_n < \infty 
        &\text{in diesem Fall ist die 
        Reihe nach oben beschränkt.}\\
        \text{oder } &\sum_{n=1}^{\infty}
        a_n = +\infty &\text{nach oben 
        unbeschränkt.}
	\end{align*}
\end{karte}
\begin{karte}{Cauchy-Kriterium für Reihen}
    Seien \(a_n \in\R, n\in\N \). Dann gilt: 
    \[ \sum_{n=1}^{\infty} a_n 
    \text{ konvergiert } \Leftrightarrow 
    \forall \varepsilon > 0 \;\exists 
    K \in \N : \abs{\sum_{j=n+1}^{m}a_j} 
    < \varepsilon \quad \forall m>n\geq K.\]
\end{karte}
\begin{karte}{Geometrische Reihe}
    Sei \(|q|<1\).
    \[ \Rightarrow \sum_{n=0}^{\infty} q^n 
    = \frac{1}{1-q}. \]
\end{karte}
\begin{karte}{Teleskopierende Summe}
    \[\sum_{n=2}^{\infty}\frac{1}{n(n-1)}=1 \]
\end{karte}
\begin{karte}{Eigenschaft der Folge einer konvergenten Reihe}
    Wenn die Reihe \(\sum_{n=1}^{\infty} 
    a_n\) konvergent, so ist \({(a_n)}_n\) 
    eine Nullfolge.\\
    Analog für \( \R^d \).
\end{karte}
\begin{karte}{Rechenregeln für unendliche Reihen}
    Sind \( \sum_{n=1}^{\infty}a_n, 
    \sum_{n=1}^{\infty}b_n \) konvergente 
    (reelle Reihen), so konvergiert auch \\
    \( \sum_{n=1}^{\infty}(\lambda a_n 
    + \mu b_n) \; \forall \lambda, 
    \mu \in\R \) und 
    \[ \sum_{n=1}^{\infty} 
    (\lambda a_n + \mu b_n) = \lambda 
    \sum_{n=1}^{\infty}a_n + \mu 
    \sum_{n=1}^{\infty}b_n. \]
\end{karte}
\begin{karte}{Cauchyscher Verdichtungssatz}
    Sei \({(a_n)}_n\) monoton fallende 
    Nullfolge. Dann gilt
    \[\sum_{n=1}^{\infty} a_n 
    \text{ konvergiert }\]
    \[\Leftrightarrow \text{ die 
    \gqq{verdichtete} Reihe } 
    \sum_{n=0}^{\infty}2^n a_{2^n} 
    = a_1 + 2a_2 + 4a_4 + 8 a_8 
    + \dots \text{ konvergiert}.\]
\end{karte}
\begin{karte}{Reihe \( \frac{1}{n^p} \)}
    \[ \sum_{n=1}^{\infty} 
    \frac{1}{n^p} \text{ konv. } 
    \Leftrightarrow p>1. \]
    \[ \sum_{n=1}^{\infty} \frac{1}{n^2} 
    = \frac{\pi^2}{6}. \]
\end{karte}
\begin{karte}{Leibnizkriterium}
    Ist \( {(a_n)}_n \) eine monoton 
    fallende Nullfolge, so konvergiert 
    die alternierende Reihe
    \[ a_1 - a_2 + a_3 - a_4 + \cdots 
    = \sum_{n=1}^{\infty} {(-1)}^{n+1} a_n. \]
\end{karte}
\begin{karte}{Absolute Konvergenz}
    Eine Reihe \( \sum_{n=1}^{\infty} a_n \) 
    heißt absolut konvergent, falls 
    \( \sum_{n=1}^{\infty} \abs{a_n} \) 
    konvergiert,\\ 
    d.\ h.\  \(\sum_{n=1}^{\infty} \abs{a_n} < \infty \).\\
    Analog für \(\R^d\).
\end{karte}
\begin{karte}{Dreiecksungleichung für Reihen}
    Eine absolut konvergente Reihe 
    \( \sum_{n=1}^{\infty}a_n, a_n \in\R \) 
    ist konvergent, und
\[ \abs{ \sum_{n=1}^{\infty}a_n }
    \leq \sum_{n=1}^{\infty}\abs{a_n}.\]
    Analog für \( \R^d \).
\end{karte}
\begin{karte}{Majorantenkriterium}
    Wir nennen eine Reihe 
    \( \sum_{n=0}^{\infty} c_n \) 
    eine Majorante, von \( \sum_{n=0}^{\infty}
    a_n, a_n\in\R \), falls \( \abs{a_n} 
    \leq c_n \) für fast alle \(n\).
    Hat die Reihe \( \sum_{n=0}^{\infty}a_n \) 
    eine konvergente Majorante, so konvergiert 
    \( \sum_{n=0}^{\infty}a_n \) absolut und 
    ist somit auch konvergent.
\end{karte}
\begin{karte}{Minorantenkrium}
    Wenn \( 0 \leq a_n \leq b_n \) für fast alle \( n \) 
    gilt, dann gilt:\\
    Divergiert \( \sum_{n=1}^\infty a_n \), so 
    divergiert auch \( \sum_{n=1}^\infty b_n \).
\end{karte}
\begin{karte}{Quotientenkriterium}
    Sei \( \sum_{n=0}^{\infty} a_n \) Reihe, 
    \( a_n\neq 0 \), und es gebe ein \(q\) mit 
    \( 0<q<1 \), sodass
    \[ \frac{\abs{a_{n+1}}}{\abs{a_n}} 
    \leq q \text{ für fast alle }n. \]
    Dann ist \( \sum_{n=0}^{\infty} a_n \) 
    absolut konvergent.\\
    Falls \( q > 1 \) ist, so divergiert die 
    Reihe.
\end{karte}
\begin{karte}{Wurzelkriterium}
    \( \sum_{n}a_n \) Reihe mit 
    \[ \limsup\limits_{n\rightarrow\infty} 
    \abs{a_n}^{\frac{1}{n}} 
    = \limsup\limits_{n\rightarrow\infty} 
    \sqrt[n]{\abs{a_n}} < 1 
    \Rightarrow \sum_n a_n 
    \text{ konv.\ abs.}.\]
    Ist \( \limsup\limits_{n\rightarrow\infty} 
    \abs{a_n}^{\frac{1}{n}} > 1 \), 
    so ist die Reihe divergent.
\end{karte}
\begin{karte}{\gqq{Mutter aller Konvergenzkriterien}}
    Sei \( \sum_n a_n \) Reihe mit 
    \(a_n \neq 0 \) für fast alle \(n\). 
    Dann gilt:
    \[ \sum_n a_n \text{ konv.\ abs.\ } 
    \Leftrightarrow \exists c_n > 0, 
    \sum_n c_n < \infty \text{ und } 
    \frac{|a_{n+1}|}{|a_n|} \leq 
    \frac{c_{n+1}}{c_n} 
    \text{ für fast alle }n. \]
\end{karte}

\end{document}