\documentclass[main.tex]{subfiles}

\begin{document}
\section*{Ableitung}
\begin{karte}{Definition Ableitung}
    \( \abb{f}{I}{\R} \) (oder \( \C, \R^d \)) 
    hat in \( x_0 \in I \) die Ableitung \( a\in \R \) 
    (oder \( \C, \R^d \)), falls 
    \[ \limesx{x}{x_0} \underbrace{\frac{f(x) - f(x_0)}{x - x_0}}_{
        \text{Differenzenquotient}
    } = a. \; (*) \]
    Notation: \( f'(x_0) := a \).\\
    Nennen \(f\) differenzierbar (diffbar) in \(x_0\), falls 
    es so ein \(a\) mit \((*)\) gibt, falls also der Grenzwert 
    in \((*)\) existiert.\\
    Alternativ, \( x = x_0 + h \Rightarrow x - x_0 = h \)
    \[ \limesx{h}{0} \frac{f(x_0 + h) - f(x_0)}{h} = a =: f'(x_0) \]    
\end{karte}
\begin{karte}{Definition Ableitung im \(\R^d\)}
    Konvergenz bezüglich Euklidischer Norm in \( \R^d 
    \Leftrightarrow \) Konvergenz der Koordinaten.
    Die Funktion \( f : I \rightarrow \R^d \) ist in \(x_0\) 
    differenzierbar \( \Leftrightarrow f = (f_1,f_2,\ldots,f_d) \), 
    alle \( f_j : I \rightarrow \R \) sind in \( x_0 \) 
    differenzierbar.
\end{karte}
\begin{karte}{Definiton Ableitungsfunktion}
    \( f : I \rightarrow \R^d \) (\( \C, \R^d \)) heißt 
    differenzierbar (diffbar) auf \(I\) (einfach differenzierbar), 
    falls \(f\) in jedem Punkt \( x_0 \in I \) differenzierbar ist. \\
    Die hierdurch gegebene Funktion 
    \[ f': I \rightarrow \R^d, x_0 \mapsto f'(x_0) 
    = \limesx{x}{x_0} \frac{f(x) - f(x_0)}{x - x_0} \]
    heißt Ableitungsfunktion oder kurz Ableitung.
\end{karte}
\begin{karte}{Zusammenhang Differenzierbarkeit und Stetigkeit}
    \( I \) offenes Intervall \( f : I \rightarrow \R^d \) 
    differenzierbar in \( x_0 \in I \Rightarrow f \) ist stetig 
    in \(x_0\).
\end{karte}
\begin{karte}{Differentiationsregeln (einfach)}
    Seien \( f, g: I \rightarrow \R^d \) differenzierbar in \(x_0\)
    \[ \Rightarrow \alpha f + \beta g, \alpha, \beta \in \R 
    \text{ ist in } x_0 \text{ differenzierbar.} \]
    \[ (\alpha f + \beta g)'(x_0) = \alpha f'(x_0) 
        + \beta g'(x_0) \]\\
    Sind \( f, g: I \rightarrow \R \) (oder \( \C \)) in \(x_0\)
    differenzierbar, so sind auch \( f\cdot g \) und im Fall 
    \( g(x_0) \neq 0 \) \( \frac{f}{g} \) differenzierbar.
\end{karte}
\begin{karte}{Produktregel}
    Sind \( f, g: I \rightarrow \R \) (oder \( \C \)) in \(x_0\)
    differenzierbar
    \[ (f \cdot g)'(x_0) = f'(x_0) \cdot g(x_0) 
    + f(x_0) \cdot g'(x_0) \]
\end{karte}
\begin{karte}{Quotientenregel}
    Sind \( f, g: I \rightarrow \R \) (oder \( \C \)) in \(x_0\)
    differenzierbar
    \[ \left( \frac{f}{g} \right)'(x_0) 
    = \frac{ f'(x_0) g(x_0) - f(x_0) g'(x_0) }{ {g(x_0)}^2 }. \]
    \[ \left(\frac{Z}{N}\right)' = \frac{NAZ - ZAN}{N^2} \]
    \( NAZ = \) Nenner mal Ableitung Zähler\\
    \( ZAN = \) Zähler mal Ableitung Nenner.
\end{karte}
\begin{karte}{Kettenregel}
    Seien \( f: I \rightarrow \R, g : J \rightarrow \R, f(I) \subset J \). 
    Ist \( f \) in \( x_0 \in I \) differenzierbar und \(g\) in 
    \( y_0 := f(x_0) \) differenzierbar, so ist auch 
    \( g \circ f : I \rightarrow \R, (g \circ f)(x) = g(f(x)) \) in \(x_0\) 
    differenzierbar und 
    \[ (g \circ f)'(x_0) = g'(f(x_0)) \cdot f'(x_0) \]
\end{karte}
\begin{karte}{Differenzierbarkeit der Umkehrfunktion}
    Sei \( f: I \rightarrow \R \) streng wachsend und stetig auf 
    dem Intervall \(I\) mit Endpunkten \( a < b \). Dann gilt 
    \begin{enumerate}
        \item \( J = f(I) \) ist ein Intervall mit Endpunkten 
        \[ \alpha := \limesx{x}{a +} f(x) 
        < \beta := \limesx{x}{b -} f(x) \]
        \item Die Umkehrfunktion \( g: J = f(I) \rightarrow I \) ist 
        ebenfalls stetig und streng wachsend.
        \item \( \limesx{y}{\alpha +} g(y) = a, 
        \limesx{y}{\beta -} g(y) = b \)
    \end{enumerate}
    Zudem ist \( I \) ein offenes Intervall, so ist auch \( J = f(I) \) 
    ein offenes Intervall und ist \(f\) in \(x_0\) differenzierbar mit 
    \( f'(x) \neq 0 \), so ist \(g\) differenzierbar in \( y_0 = f(x_0) \) 
    mit 
    \[ g'(y_0) = \frac{1}{f'(g(y_0))} = \frac{1}{f'(x_0)}. \]
\end{karte}
\begin{karte}{Einseitige Grenzwerte monotoner Funktionen}
    Sei \( h: I \rightarrow \R \) monoton wachsend, \(I\) ein Intervall mit 
    Endpunkten \( a < b \).\\
    Dann hat \(h\) einseitige Grenzwerte. Genauer
    \[ \limesx{x}{x_0 -} h(x) = \lim\limits_{x \nearrow x_0} h(x) 
    = \sup \underset{\text{falls } x_0 > a}{\set{ h(x) 
    \; \vert \; x \in I, x < x_0 }} \]
    \[ \limesx{x}{x_0 +} h(x) = \lim\limits_{x \searrow x_0} h(x) 
    = \inf \underset{\text{falls } x_0 < b}{\set{ h(x) 
    \; \vert \; x \in I, x > x_0 }} \]
    Analoge Aussage gilt, falls \(h\) monoton fallend ist.
\end{karte}
\end{document}