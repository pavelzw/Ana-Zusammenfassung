\documentclass[main.tex]{subfiles}

\begin{document}
\section*{Starke Induktion}
\begin{karte}{Starke Induktion}
    Seien \(A(n) \) Aussagen für alle \(n \in \N \). Gilt
	\begin{enumerate}
		\item \(A(1) \) ist wahr.
        \item \(\forall \, n \in \N: \; A(1), \ldots, A(n) \) wahr 
        \(\Rightarrow A(n + 1) \) wahr.
	\end{enumerate}
	So ist ist \(A(n) \) wahr für alle \(n \in \N \).
\end{karte}
\section*{Wohlordnungsprinzip}
\begin{karte}{Wohlordnungsprinzip für \( \N \) und \( \Z \)}
    Jede nichtleere Teilmenge der 
    natürlichen Zahlen \(\N \) 
    besitzt ein kleinstes Element.\\
    Jede nichtleere, nach unten 
    beschränkte Teilmenge in \( \Z \) 
    besitzt ein kleinstes Element.
\end{karte}
\begin{karte}{\(\Q\) ist dicht in \(\R\)}
    Seien \(a, b \in \R, a < b \). 
    Dann existiert \( r \in \Q \) mit \(a < r < b \).
\end{karte}
\begin{karte}{Wurzel einer natürlichen Zahl}
    Sei \(k \in \N \), dann ist entweder 
    \(\sqrt{k} \in \N \) oder \(\sqrt{k} \in \R \setminus \Q \).
\end{karte}
\end{document}