\documentclass[main.tex]{subfiles}

\begin{document}
\section*{Potenzreihen}

\begin{karte}{Definition Potenzreihe}
    Eine Potenzreihe ist eine Reihe der Form
    \[ \sum_{n=0}^\infty a_n z^n, a_n \in\C, z \in\C. \]
    Partialsummen \( S_n(z) = \sum_{j=0}^n a_j z^j \) (Polynome).\\
    Definiere \( R:= \sup \set{ \abs{z} \;\vert \; z\in\C, 
    \sum_{n=0}^\infty a_n z^n \text{ konvergent} } \)
    (Konvergenzradius). (\( R = 0, \infty \) sind erlaubt.)\\
    Die Potenzreihe konvergiert absolut für jedes \( z\in\C \) 
    mit \( \abs{z} < R \) (d.\ h.\  \( z\in B_R(0) \)) 
    und divergiert für jedes \( \abs{z} > R \).
\end{karte}
\begin{karte}{\( \sin \), \( \cos \), eulersche Formel, \( \sinh, \cosh \)}
    \[ \cos z := \sum_{n=0}^\infty {(-1)}^n \frac{z^{2n}}{2n!}
    \qquad \sin z  := \sum_{n=0}^\infty {(-1)}^n 
    \frac{z^{2n+1}}{(2n+1)!}. \]
    \[ \cos x = \frac{e^{ix} + e^{-ix}}{2} \qquad 
    \sin x = \frac{e^{ix} - e^{-ix}}{2i} \]
    Es gilt:
    \[ e^{iz} = \cos z + i \sin z \quad \forall z\in\C. \]
    Es gilt:
    \[ \cos^2 t + \sin^2 t = 1 \quad \forall t \in \R. \]
    \[ \cosh x = \sum_{n=0}^\infty \frac{z^{2n}}{(2n)!} \qquad 
    \sinh x = \sum_{n=0}^\infty \frac{z^{2n + 1}}{(2n + 1)!}. \]
    \[ \cos x = \frac{e^{x} + e^{-x}}{2} \qquad 
    \sin x = \frac{e^{x} - e^{-x}}{2}. \]
    \[ \cosh^2 t - \sinh^2 t = 1 \quad \forall t \in \R. \]
\end{karte}
\begin{karte}{Konvergenzradius Wurzelkriterium}
    Für den Konvergenzradius \( R \) von 
    \( \sum_{n=0}^\infty a_n z^n \) gilt
    \[ R = \frac{1}{\limessup{n} \sqrt[n]{\abs{a_n}}}, \]
    wobei \(\frac{1}{0} := \infty \) und 
    \( \frac{1}{\infty} := 0 \) gesetzt wird.
\end{karte}
\begin{karte}{Konvergenzradius Quotientenkriterium}
    Für den Konvergenzradius \( R \) von 
    \( \sum_{n=0}^\infty a_n z^n \) gilt
    \[ R = \limes{n} \frac{\abs{ a_n }}{\abs{ a_{n+1} }}, \]
    wobei \(\frac{1}{0} := \infty \) und 
    \( \frac{1}{\infty} := 0 \) gesetzt wird.
\end{karte}
\begin{karte}{Indexverschiebung der Potenzreihe}
    Die Potenzreihen \( \sum_{n=0}^\infty a_n z^n \) 
    und \( \sum_{n=1}^\infty n \cdot a_n z^{n-1} \) 
    haben denselben Konvergenzradius.
\end{karte}
\begin{karte}{Maximum einer Potenzreihe}
    Konvergiert \( \sum_{n=0}^\infty a_n z^n \) 
    für ein \( z = z_1 \neq 0 \), so ist sie auf 
    der Kreisscheibe \( B_r (0)  
    = \set{ z\in\C \;\vert \; \abs{z}\leq r } \)
    beschränkt für jedes \( 0 < r < \abs{z_1} \).\\
    D.\ h.\  \( \exists \, M(r) \geq 0 \), sodass
    \[ \abs{ \sum_{n=0}^\infty a_n z^n } \leq M(r) 
    \quad \forall \, \abs{z} < r. \]
    Ang.\  \( \sum_{n=0}^\infty a_n z^n \) konvergiert 
    für ein \( z = z_1 \neq 0 \). \\
    Dann gibt es zu jedem
    \[ 0 < r < \abs{z_1} \text{ und } k\in\N_0\]
    ein \( M(r,k) > 0 \) mit
    \[ \abs{ \sum_{n=1}^\infty a_n z^n } 
    \leq M(r,k) \abs{z}^{k+1} \quad \forall \, \abs{z}\leq r. \]
\end{karte}
\begin{karte}{Nullreihensatz}
    Es gebe eine Folge \( {(z_j)}_j, z_j \in\C, z_j 
    \neq 0 \) mit \( z_j \rightarrow 0 \), sodass 
    \[ \sum_{n=0}^\infty a_n z_j^n = 0 \quad \,\forall \, j\in\N. \]
    \[ \Rightarrow a_n = 0 \quad \forall n\in\N_0. \]
\end{karte}
\end{document}