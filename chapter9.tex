\documentclass[main.tex]{subfiles}

\begin{document}

\section*{Cauchyfolgen}

\begin{karte}{Definition Cauchyfolge}
    Eine Folge \({(a_n)}_n\) heißt Cauchyfolge (kurz Cauchy), falls
    \[ \forall \varepsilon > 0 \,\exists K\in\N: 
    \abs{a_n-a_m} < \varepsilon \, \forall n,m\geq K. \]
\end{karte}
\begin{karte}{Eigenschaften Cauchyfolge}
    Jede konvergente Folge ist eine Cauchyfolge.\\
    Jede Cauchyfolge ist beschränkt.\\
    Jede Cauchy-Folge hat eine konv.\ Teilfolge.\\
    Für eine (reelle) Folge \({(a_n)}_n\) gilt:\\
    \({(a_n)}_n\) konvergiert 
    \( \Leftrightarrow {(a_n)}_n \) ist Cauchyfolge.
\end{karte}
\begin{karte}{Definition Teilfolge/Ausdünnung}
    Eine Funktion \( \sigma:\N \rightarrow\N \) 
    heißt Ausdünnung, falls \( \sigma (n+1) 
    > \sigma(n) \, \forall n\in\N \) 
    (d.\ h.\  \(\sigma \) ist streng monoton wachsend).
\end{karte}
\begin{karte}{Satz von Bolzano-Weierstraß}
    Jede beschränkte Folge hat eine konvergente Teilfolge. 
    (gilt auch in \( \R^d \))
\end{karte}

\end{document}