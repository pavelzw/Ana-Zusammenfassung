\documentclass[main.tex]{subfiles}

\begin{document}
\section*{Reihen}
\begin{karte}{Cauchykriterium für Reihen}
    Eine Reihe \( \sum_{n=1}^\infty a_n \) 
    in \( \R^d \) oder \( \C \) konvergiert
    \[ \Leftrightarrow \, \forall \, 
    \varepsilon > 0 \,\exists \, K\in\N: 
    \abs{\sum_{j=n+1}^m a_j } < \varepsilon 
    \,\forall \, m>n\geq K. \]
    \[ \Leftrightarrow \, \forall \, 
    \varepsilon > 0 \,\exists \, K\in\N: 
    \abs{\sum_{j=n+1}^{n+p} a_j } < \varepsilon 
    \,\forall \, n\geq K, p\in\N. \]
\end{karte}
\begin{karte}{Umordnung}
    Seien \( \sum_{n=1}^\infty a_n, 
    \sum_{n=1}^\infty b_n \) Reihen 
    in \( \R^d \). \\
    \( \sum_{n=1}^\infty b_n \) 
    ist eine Umordnung von 
    \( \sum_{n=1}^\infty a_n \), falls eine 
    bijektive Abbildung \\
    \( \sigma:\N\rightarrow\N \) 
    existiert mit
    \[ b_n = a_{\sigma(n)} \,\forall \, n\in\N. \]
    Sei \( \sum_n a_n \) eine absolut konvergente Reihe. 
    Dann konvergiert jede Umordnung gegen denselben Wert.
\end{karte}
\begin{karte}{Unbedingte Konvergenz}
    Eine Reihe \( \sum_n a_n \) heißt unbedingt konvergent, 
    falls jede Umordnung \( \sum_n b_n \) ebenfalls 
    konvergiert und denselben Grenzwert besitzt. 
    Andernfalls heißt \( \sum_n a_n \) bedingt konvergent.
    \[ \sum_n a_n, a_n \in\R^d \text{ ist unbedingt konvergent } 
    \Leftrightarrow \sum_n a_n \text{ ist absolut konvergent.} \]
\end{karte}
\begin{karte}{Satz von Riemann}
    Sei \( \sum_n a_n, a_n \in\R \) eine bedingt konvergente 
    Reihe. Dann existiert zu jeder Zahl \( c\in\R \) eine 
    Umordnung \( \sum_n b_n \) von \( \sum_n a_n \) mit 
    \( \sum_{n=1}^\infty b_n = c \).
\end{karte}
\begin{karte}{Cauchyprodukt}
    \( a_n, b_n \in\C \). Die Reihen 
    \( \sum_{n=0}^\infty a_n, \sum_{n=0}^\infty b_n \) 
    seien absolut konvergent.\\
    Dann ist auch \( \sum_{n=0}^\infty c_n \) mit 
    \[ c_n := \sum_{k+l=n} a_k b_l = \sum_{k=0}^n a_k b_{n-k} \]
    absolut konvergent und \( \sum_n c_n 
    = ( \sum_{k=0} a_k )( \sum_{l=0} b_l ) \) \\
    und ist \(\tau \colon \; \N_0 \rightarrow \N_0 \times \N_0, 
    n \mapsto \tau(n) = (\tau_1(n), \tau_2(n)) \) bijektiv
    \( p_n := a_{\tau_1(n)} \cdot b_{\tau_2(n)} \rightarrow 
    \sum_{n=0}^\infty p_n \) absolut konvergent \\
    und \( \sum_{n=0}^\infty p_n = \sum_{k=0}^\infty 
    a_k \cdot \sum_{l=0}^\infty b_l \).
\end{karte}
\end{document}