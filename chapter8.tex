\documentclass[main.tex]{subfiles}

\begin{document}

\section*{Monotone Konvergenz}
\begin{karte}{Definition monoton wachsend, fallend}
    Eine Folge \({(a_n)}_n\) heißt \\
    monoton wachsend, falls 
    \(a_n\leq a_{n+1} \quad \forall n\in\N \).\\
    monoton fallend, falls \(a_{n+1} \leq 
    a_n \quad \forall n\in\N \).\\
    Ähnlich: monoton wachsend (fallend) 
    für fast alle \(n\in\N \), falls \( K\in\N \) 
    existiert mit \( a_n\leq a_{n+1} \, 
    \forall \, n\geq K \) (bzw. \( a_{n+1}
    \leq a_n \, \forall n\geq K \)).\\
	Ist \\
    \( a_n<a_{n+1} \, \forall n\in\N \), 
    so heißt \(a_n\) streng monoton wachsend.\\
    \( a_{n+1}<a_n \, \forall n\in\N \), 
    so heißt \(a_n\) streng monoton fallend.
\end{karte}
\begin{karte}{Monotone Konvergenz}
    Jede monoton wachsende, nach oben beschränkte 
    Folge ist konvergent. Jede monoton fallende, 
    nach unten beschränkte Folge ist konvergent.
\end{karte}
\begin{karte}{Folge und Reihe für \(e\)}
    \[ e^x = \limes{n} {\left(1+\frac{x}{n}\right)}^n. \]
    \[ e^x = \sum_{n=0}^\infty \frac{x^n}{n!}. \]
    \( e \) ist irrational.
\end{karte}
\begin{karte}{Intervallschachtelungsprinzip}
    Seien \( a_n\leq b_n, I_n := [a_n,b_n] \) 
    abgeschlossene Intervalle und 
    \[I_{n+1} \subset I_n \quad \forall n\in\N \]
    sowie \( |I_n| := b_n - a_n 
    \overset{n\rightarrow\infty}{\longrightarrow} 0 \).\\
	Dann besteht \(\bigcap_{n\in\N} I_n\) aus genau einem Punkt.\\
\end{karte}
\begin{karte}{\(k\)-adische Darstellung}
    \(k\in\N, k\geq Z\) und \(x\in\R \). 
    Dann gibt es \(z_0\in \Z \) und 
    \(l_j \in \{0,1,\ldots,k-1\} \) derart, 
    dass \(x = z_0 + \limes{n} \sum_{j=1}^{n} 
    l_j k^{-j} = z_0 + \sum_{j=1}^{\infty} l_j k^{-j} \).\\
    \(Z_0 := \lfloor x \rfloor := \min(p\in\Z, p>x) -1 
    = \max(q\in\Z, q\leq x). \) \\
	\( 0\leq x-\lfloor x\rfloor <1. \) \\
\end{karte}

\end{document}