\documentclass[main.tex]{subfiles}

\begin{document}
\section*{Grenzwerte für Funktionen}
\begin{karte}{Definition punktierter \(\delta\)-Ball um \(x_0\)}
    Für \( x_0 \in \R^n, \delta > 0 \) ist
    \[ \dot{B}_\delta(x_0) = \set{ x \in \R^n \;\colon \; 
    0 < \abs{x-x_0} < \delta } = B_\delta (x_0) \setminus 
    \set{x_0} \]
    der punktierte \( \delta \)-Ball um \( x_0 \).
\end{karte}
\begin{karte}{Definition Häufungswert/Berührpunkt einer Menge}
    Gegeben \( D \subset \R^n, x_0 \in \R^n \) heißt 
    Häufungswert (von \( D \)), falls 
    \[ \forall \, \delta > 0 : \underbrace{ 
    \dot{B}_\delta(x_0) \cap D}_{ = \set{ x\in D \;\colon \;
    0 < \abs{ x - x_0 } < \delta } } \neq \emptyset. \]
    Andere Definition:
    \( x_0 \) ist der Häufungswert von \( D \) (Berührpunkt)
    \[ \Leftrightarrow \exists \text{ Folge } {(x_n)}_n \subset D, 
    x_n \neq x_0, x_n \rightarrow x_0. \]
\end{karte}
\begin{karte}{Definition Konvergenz einer Funktion}
    Die Funktion \( \abb{f}{D}{\R^m} \) konvergiert für 
    \( x\rightarrow x_0 \) gegen \( \alpha \in \R^m \), falls
    \[ \forall \,\varepsilon > 0 \,\exists \, \delta > 0 
    \;\colon \; \abs{ f(x) - \alpha } < \varepsilon \,\forall \, 
    x \in D\text{ mit } 0 < \abs{x-x_0} < \delta. \]
    Notation:
    \[ \limesx{x}{x_0} f(x) = \alpha \text{ oder } f(x) 
    \rightarrow \alpha \text{ für } x\rightarrow x_0 \]
    Für die Existenz und den Wert von \( \limesx{x}{x_0} f(x) \)
    ist es egal, ob die Funktion in \( x_0 \) definiert ist oder welchen 
    Wert sie dort hat. \\
    Ist \( f: D \rightarrow \R^m, D \leq \R \) und \( x_0 \in D \),\\
    \( D \cap (x_0 - \delta, x_0 + \delta) \neq \emptyset \;\forall \, 
    \delta > 0 \). Dann gilt: 
    \[ \limesx{x}{x_0} f(x) = a \Leftrightarrow
    \limesx{x}{x_0-} f(x) = a \text{ und } \limesx{x}{x_0+} 
    f(x) = a. \]
\end{karte}
\begin{karte}{Stetigkeit und Grenzwert}
    Sei \( D \subset \R^n, \abb{f}{D}{\R^m}, x_0 \in D \) 
    Häufungspunkt von \(D\). Dann sind äquivalent
    \begin{enumerate}
        \item \( \limesx{x}{x_0} f(x) = f(x_0) \)
        \item \( f \) ist stetig in \( x_0 \).
    \end{enumerate}
\end{karte}
\begin{karte}{Grenzwert einer Funktion bei \( \pm \infty \)}
    Sei \( D \subset \R \) nach oben unbeschränkt und 
    \( f:D\rightarrow \R^m \). Dann gilt 
    \[ \limes{x} f(x) = \alpha \in \R^m, \]
    falls
    \[ \forall \,\varepsilon > 0 \,\exists \, K\in\R \text{ mit } 
    \abs{f(x) - \alpha} < \varepsilon \,\forall \, x \in D 
    \text{ mit } x > K. \]
    Analog: \( \limesx{x}{-\infty} f(x). \)
\end{karte}
\begin{karte}{Einseitiger Grenzwert}
    Ist \( D \subset \R, f : D\rightarrow \R^m \) und \( x_0 \in D \)
    mit \( (x_0, x_0 + \delta) \cap D \neq \emptyset \) für alle 
    \( \delta > 0 \). Dann ist \( \limesx{x}{x_0+} f(x) = \alpha \) 
    (rechtsseitiger Grenzwert), falls
    \[ \forall \,\varepsilon > 0 \,\exists \, \delta > 0 \;\colon \;
    \abs{ f(x) - \alpha } < \varepsilon \,\forall \, \underbrace{ x\in D \cap 
    (x_0, x_0 + \delta) }_{ \text{oder: } x\in D \text{ mit } 
    x_0 < x < x_0 + \delta }. \]
    Ist \( (x_0 - \delta, x_0) \cap D \neq \emptyset \,\forall \, 
    \delta > 0 \), so ist \( \limesx{x}{x_0-} f(x) = \alpha \), falls
    \[ \forall \, \varepsilon > 0 \,\exists \, \delta > 0 \;\colon \; 
    \abs{f(x) - \alpha} < \varepsilon \,\forall \, x \in D \cap 
    (x_0 - \delta, x_0). \]
\end{karte}
\begin{karte}{Grenzwerte von Funktionen mit Folgen}
    Sei \( D \subseteq \R^n \), \(x_0 \) Häufungspunkt von 
    \( D \), \( f: D \rightarrow \R^m \). Für \( a\in \R^m \) 
    sind äquivalent:
    \begin{enumerate}
        \item \( f(x) \rightarrow a \) für \( x\rightarrow x_0 \)
        \item \( \limes{k} f(x_k) = a \,\forall \, {(x_k)}_k 
        \subseteq D \setminus \{x_0\} \) mit \( x_k \rightarrow x_0 \).
    \end{enumerate}
\end{karte}
\begin{karte}{Rechenregeln für Grenzwerte}
    Sei \( x_0 \in D \subset \R^n \) Häufungspunkt von \( D \). 
    Dann gilt:
    \begin{enumerate}
        \item 
        \( \lambda f(x) + \mu g(x) \rightarrow \lambda a + \mu b
        \text{ für } x \rightarrow x_0. \)
        \item 
        \( f(x) g(x) \rightarrow a b \text{ für } 
        x\rightarrow x_0. \) \\
        Ist \( b\neq 0 \), so ist \( \forall \, x\in 
        D \cap B_\delta(x_0) : g(x) \neq 0 \) und 
        \[ \frac{f(x)}{g(x)} \rightarrow \frac{a}{b} \text{ für } 
        x\rightarrow x_0. \]
        \item Sind \( f: D \rightarrow \R^m, f(D) \subseteq E 
        \subseteq \R^m, g : E \rightarrow \R^l \), gilt 
        \( f(x) \rightarrow y_0 \) für \( x\rightarrow x_0 \) und
        ist \(g\) stetig in \( y_0 \), so folgt
        \[ (g\circ f)(x) = g(f(x)) \rightarrow g(y_0) \text{ für }
        x \rightarrow x_0. \]
        \item Ist \( f : D \rightarrow (0, \infty) \), so ist 
        \( \limesx{x}{x_0} f(x) = 0 \) äquivalent zu 
        \( \limesx{x}{x_0} \frac{1}{f(x)} = \infty \).
    \end{enumerate}
\end{karte}
\begin{karte}{Stetigkeit von Potenzreihen}
    Sei \( f(z) := \sum_{n=0}^\infty a_n z^n \) eine 
    Potenzreihe mit Konvergenzradius \( \rho > 0 \). 
    Dann ist \( f: B_\rho(0) \rightarrow \C \) stetig.
\end{karte}
\end{document}