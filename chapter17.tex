\documentclass[main.tex]{subfiles}

\begin{document}
\section*{Eigenschaften stetiger Funktionen}
\begin{karte}{Definition Gleichmäßige Stetigkeit}
    Eine Funktion \( f: D \rightarrow \C \) 
    heißt gleichmäßig stetig, falls
    \[ \forall \, \varepsilon > 0 \,\exists \, 
    \delta > 0 \,\forall \, z_1, z_2 \in D \text{ mit } 
    \abs{z_1 - z_2} < \delta \Rightarrow 
    \abs{f(z_1) - f(z_2)} < \varepsilon. \]
    Jede gleichmäßig stetige Funktion ist stetig.
\end{karte}
\begin{karte}{Satz von Heine}
    Sei \( D \subset \C \) abgeschlossen und beschränkt
    und \( f: D \rightarrow \C \) stetig. \\
    Dann ist \(f\) gleichmäßig stetig. \\
    Alle Polynome, \( \exp, \sin, \cos \) sind somit
    auf abgeschlossenen und beschränkten Teilmengen
    von \( \C \) gleichmäßig stetig. Insbesondere 
    gilt dies auf allen Intervallen \( [a,b], 
    a<b \in\R \).
\end{karte}
\begin{karte}{Definition Folgenkompaktheit}
    Sei \( D \subset \R \) oder \( D \subset \C \)
    heißt folgenkompakt oder einfach kompakt, wenn
    jede Folge \( {(a_n)}_n \subset D \) eine
    konvergente Teilfolge \( {(a_{\sigma(n)})}_n \)
    besitzt mit \( \limes{n} a_{\sigma(n)} \in D \), 
    wobei \( \sigma : \N \rightarrow \N \) eine 
    Verdünnung ist.
\end{karte}
\begin{karte}{Heine-Borel}
    \( D \subset \C \) ist beschränkt und abgeschlossen, 
    wenn \( {(z_n)}_n \subset D \) eine konvergente
    Teilfolge besitzt, deren Grenzwert wieder in 
    \( D \) liegt.
\end{karte}
\begin{karte}{Satz vom Minimum und Maximum einer Funktion}
    Sei \( D \subset \C \) abgeschlossen und 
    beschränkt und \( f : D \rightarrow \R \) eine
    stetige Funktion.\\
    Dann nimmt \(f\) sein Minimum und sein Maximum 
    auf \(D\) an, d.\ h.\ es gibt ein \( z_{\min} \)
    und \( z_{\max} \) mit
    \begin{align*}
        f(z_{\min}) &= \inf f(D) &= \min f(D) \\
        f(z_{\max}) &= \sup f(D) &= \max f(D)
    \end{align*}
    Insbesondere gilt 
    \[ \abs{f(z)} \leq \max \left( \abs{ f(z_{\min)} }, 
    \abs{ f(z_{\max}) } \right) \; \forall \, z\in D. \]
\end{karte}
\begin{karte}{Stetige Fortsetzung}
    Es sei \( D \subset \C \) nicht abgeschlossen, 
    \( f: D\rightarrow \C \) und \( z_0 \in \bar{D}
    \setminus D \) und \( w_0 \in \C \) seien gegeben.\\
    Wenn Grenzwert \( \limesx{z}{z_0} f(z) = w_0 \)
    existiert, dann heißt die Funktion 
    \[ \bar{f} : \bar{D} := D \cup \set{z_0} 
    \rightarrow \C \]
    \[ \bar{f}(z) := 
    \begin{cases}
        f(z), &z\in D \\
        w_0, &z = z_0
    \end{cases} \]
    stetige Fortsetzung von \(f\) auf \(z_0\). \\

    Sei \( f: D\rightarrow \C \) gleichmäßig stetig. 
    Dann hat \(f\) eine eindeutige stetige Fortsetzung 
    \( \bar{f} : \bar{D} \rightarrow \C \). Diese ist 
    auch gleichmäßig stetig.
\end{karte}
\begin{karte}{Zwischenwertsatz}
    Eine stetige Funktion \( f: [a,b] \rightarrow \R \) 
    nimmt jeden Wert \( \gamma \) zwischen \( f(a) \) 
    und \( f(b) \) an mindestens einer Stelle 
    \( c\in [a,b] \) an, d.\ h.\ es gibt ein 
    \( c \in [a,b] : f(c) = \gamma \). \\
    Ist \( f(a) \neq f(b) \) und liegt \( \gamma \) strikt 
    zwischen \( f(a) \) und \( f(b) \), dann existiert 
    \( c \in (a,b) \), sodass \( f(c) = \gamma \).
\end{karte}
\begin{karte}{Nullstellensatz}
    Sei \( f : [a,b] \rightarrow \R \) stetig 
    mit \( f(a) \leq 0 \leq f(b) \). Dann existiert 
    ein \( c \in [a,b] \) mit \(f(c) = 0\), d.\ h.\ 
    \(f\) hat eine Nullstelle in \( [a,b] \).\\
    Ist \( f(a) < 0 < f(b) \), so ist \( c \in (a,b) \). \\
    Sei \( f: [a,b] \rightarrow \R \) stetig und 
    \( f(a) \cdot f(b) \leq 0 \), dann existiert 
    ein \( c \in [a,b] \) mit \( f(c) = 0 \).
\end{karte}
\end{document}