\documentclass[main.tex]{subfiles}

\begin{document}
\section*{Körper}
\begin{karte}{Körperaxiome}
    Sei \( (\K, +, \cdot ) \) ein Körper.
    \begin{description}
        \item[Kommutativgesetze] \( \forall a,b\in \K: a + b = b + a, a \cdot b = b \cdot a \).
        \item[Assoziativgesetze] \( \forall a,b,c \in \K: 
        a + (b + c) = (a + b) + c, a \cdot (b \cdot c) = (a \cdot b) \cdot c \).
        \item[Distributivgesetz] \( \forall a,b,c\in \K: 
        a \cdot (b + c) = a \cdot b + a \cdot c \).
        \item[Neutrale Elemente] 
        \( \exists 0 \in \K: a + 0 = 0 + a = a \,\forall a \in \K \).\\
        \( \exists 1 \in \K: a \cdot 1 = 1 \cdot a = a \,\forall a \in \K \).
        \item[Inverse Elemente] 
        \( \forall a \in \K \;\exists -a \in \K: a + (-a) = 0. \).\\
        \( \forall a \in \K \setminus \set{0} \, \exists a^{-1} \in \K: a \cdot a^{-1} = 1. \)
    \end{description}
\end{karte}
\begin{karte}{Dreiecksungleichung}
   Dreiecksungleichung: 
   \[ \abs{a + b} \leq \abs{a} + \abs{b}. \]

   Umgekehrte Dreiecksungleichung: 
   \[ \abs{ \abs{a} - \abs{b} } \leq \abs{ a - b }. \]
\end{karte}
\begin{karte}{Geometrisch-Arithmetische Ungleichung}
    \[ ab \leq {\left( \frac{a + b}{2} \right)}^2. \]
\end{karte}
\subsection*{Schranken}
\begin{karte}{Definition Schranke}
    \( \gamma \) heißt 
    obere Schranke von \(A\), falls \( \forall a \in A : a \leq \gamma \).\\
    Schreibe auch \( A \leq \gamma \).\\
    Analog für untere Schranken.
    \( A \) heißt beschränkt, wenn \(A\) nach oben und unten beschränkt ist.
\end{karte}
\begin{karte}{Definition Supremum, Infimum}
    Die kleinste obere Schranke heißt Supremum.
    \[ \alpha = \sup A \Leftrightarrow \alpha \geq A \wedge 
    \forall \varepsilon > 0 \,\exists a \in A: \alpha - \varepsilon < a. \]
    Die größte untere Schranke heißt Infimum.
    \[ \beta = \inf B \Leftrightarrow \beta \geq B \wedge 
    \forall \varepsilon > 0 \,\exists b \in B: \beta - \varepsilon < b. \]
\end{karte}
\subsection*{Das Vollständigkeitsaxiom}
\begin{karte}{Vollständigkeitsaxiom}
    Ein angeordneter Körper \( \K \) erfüllt das Vollständigkeitsaxiom, falls gilt:\\
    Jede nichtleere, nach oben beschränkte Teilmenge von \( \K \) hat ein Supremum.
\end{karte}
\subsection*{Die natürlichen Zahlen \( \N \)}
\begin{karte}{Definition induktive Menge}
    Eine Teilmenge \( M \subset \R \) heißt \textit{induktiv}, falls gilt:
    \begin{itemize}
        \item \( 1 \in M \).
        \item Aus \( n \in M \) folgt \( n + 1 \in M \).
    \end{itemize}
\end{karte}
\begin{karte}{Definition natürliche Zahlen}
    \[ \N := \bigcap_{\substack{M\subset \R \\ \text{induktiv}}} M. \]
\end{karte}
\begin{karte}{Archimedisches Prinzip}
    \begin{itemize}
        \item \( \N \) ist nicht nach oben beschränkt.
        \item \( \forall x \in \R \) mit \( x > 0 \, \exists n \in \N: \frac{1}{n} < x. \)
    \end{itemize}
\end{karte}
\begin{karte}{Laufindexverschiebung}
    \[ \sum_{k=m}^n a_k = \sum_{j=m-1}^{n-1} a_{j+1} = \cdots = \sum_{l=0}^{n-m} a_{l+m}. \]
    Beispiel:
    \[ \sum_{i=1}^n 2^i = \sum_{i=0}^{n-1} 2^{i+1} = 2\sum_{i=0}^{n-1} 2^i. \]
\end{karte}
\begin{karte}{Bernoullische Ungleichung}
    Sei \( x \in \R, x \geq -1, n \in \N_0 \). Dann gilt 
    \[ {(1+x)}^n \geq 1 + nx. \]
    Es gilt \gqq{\(>\)}, falls \(n > 1, x \neq 0\).
\end{karte}
\begin{karte}{Geometrische Summe}
    \[ \sum_{k=0}^n x^k = \frac{1 - x^{n+1}}{1 - x}. \]
\end{karte}
\end{document}